\input zrcadlo % charger la macro pour la composition parall`ele
% param`etres d’empagement :
\newdimen\Hsize \Hsize=170mm % largeur du rectangle d’empagement (avec deux
% colonnes)
\hsize=82mm % largeur d’une colonne
\vsize=706pt % hauteur du rectangle d’empagement (59 lignes) \hoffset=-1in \advance\hoffset by20mm % marge de gauche \voffset=-1in \advance\voffset by20mm % marge de te^te \ifx\pdfoutput\undefined \else % pour pdfTeX
   \pdfpagewidth=210mm % largeur de la page
   \pdfpageheight=297mm % hauteur de la page
\fi
\clubpenalty=10000 % empe^cher des orphelins \widowpenalty=10000 % empe^cher des veuves
\brokenpenalty=0 % page finie par un mot coupé est acceptable \parskip=0pt % pas d’espace avant les paragraphes \baselineskip=12pt plus 1pt % interligne élastique
% macro pour charger des polices avec un espacement d’un tiers de cadratin :
\def\myfont#1 at{\def\temp{#1}\afterassignment\domyfont \dimen0=}
\def\domyfont{\expandafter\dodomyfont\temp\relax}
\def\dodomyfont#1#2\relax{\font#1#2 at\dimen0 \fontdimen6#1=\dimen0
   \fontdimen2#1=.333333\dimen0 \fontdimen3#1=.166667\dimen0
\fontdimen4#1=.111111\dimen0 \fontdimen7#1=.111111\dimen0 } \myfont\tenrm=utmr8p at10pt % romain
\myfont\tenbf=utmb8p at10pt % gras
\myfont\smallrm=utmr8p at6pt % police pour le numéro de verset \myfont\bigbf=uhvb8p at12pt % police pour le numéro de chapitre \myfont\tengrm=opbgr at10pt % grec
\rm % police par défaut
\frenchspacing % espacement a` la franc ̧aise \parindent=1.5em % renfoncement d’alinéa \tolerance=750
% définitions spécifiques au texte grec : \def\nastavlevy{\tengrm \language=\greek
\grcode} % mettre \lccode `a des valeurs approprieés au grec (macro % définie dans gcsplain)
% définitions spécifiques au texte latin : \def\nastavpravy{\language=\latin}
% chapitre :
\def\cap#1{\par\vskip0pt\indent{\bigbf #1\rm\enspace}\ignorespaces}
% verset (le numéro de verset sera imprimé en exposant avant le verset) : \def\sec#1{\vers \raise.85ex\hbox{\smallrm #1}\ignorespaces} \def\secx#1{\vers\ignorespaces} % verset manquant
% `a la fin de chaque fichier source, il y a une commande \konec ; on
% l’emploiera pour signaler la derni`ere page `a la routine de sortie (sur la
% derni`ere page, on veut, en effet, ajuster les interlignes de la plus courte % des deux colonnes de sorte qu’elle ait la m^eme hauteur que l’autre
% colonne) :
\def\konec{\penalty10001\penalty-10001 }
\newbox\colonnedegauche % pour stocker la colonne de gauche
% routine de sortie :
\output={\if L\strana \global\setbox\colonnedegauche=\pagebody
   \else \setbox255=\pagebody \preparerladernierepage \imprimercolonnes \fi}
   \def\preparerladernierepage{%
\ifnum\lastpenaltyL=10001 \ifnum\lastpenaltyP=10001 % derni`ere page
% mettre les deux colonnes dans des boı^tes verticales ayant leur hauteur % naturelle respective : \setbox\colonnedegauche=\vbox{\unvbox\colonnedegauche} \setbox255=\vbox{\unvbox255}
      % comparer les hauteurs respectives des colonnes :
      \ifdim\ht\colonnedegauche<\ht255
         \setbox\colonnedegauche=\vbox to\ht255{\unvbox\colonnedegauche}
      \else
         \setbox255=\vbox to\ht\colonnedegauche{\unvbox255}
   \fi\fi\fi}
\def\imprimercolonnes{\shipout\vbox{
      \makeheadline
      \vbox to \vsize{\hbox to\Hsize{\leftline{\box\colonnedegauche}\hfil
         \vrule depth3.5pt width.5pt
         \hfil\leftline{\box255}}\kern-\prevdepth\vfil}
      \makefootline}
   \advancepageno}
% afin que la ligne de te^te et celle de pied s’étendent sur la largeur des % deux colonnes, il faut redéfinir :
\def\makeheadline{\vbox to0pt{\vskip-22.5pt
   \hbox to\Hsize{\vbox to8.5pt{}\the\headline}\vss}\nointerlineskip}
\def\makefootline{\baselineskip=24pt \lineskiplimit=0pt
   \hbox to\Hsize{\the\footline}}
% étant donné la composition en deux colonnes, il n’est pas nécessaire de % commencer sur une page de gauche :
\let\novalevastrana=\relax
\zrcadli mc-g mc-l % commencer la composition parall`ele \bye % fin